%% This file is to be included by latex in wip.tex
%
% The User Interface
%
\mylabel{c:iface}
\myfile{iface.tex}

There are, at present, two ways to create plots with \wip: interactively
or by reading plot command files from the command line.
Command line plotting is an efficient way to spool plots to a printer or
to load plot command files and macros when starting \wip.
Command line plotting will be discussed more fully in Appendix~\ref{a:cmdline}.
Interactive plotting consists of typing the
commands that generate the plot, one at a time, and seeing the results.
This chapter describes the user interface used by \wip.

When \wip\ is started, certain commands and parameters
are defined by the program as well as by the user
(see Appendix~\ref{a:wipinit}\index{wipinit@\wipinit}
for details on creating your own initialization file).
The user is then presented with the standard \wip\ prompt (\wip\verb+>+)
and commands may be then entered one per line.
Command lines in \wip\ consist of one or more ``words'' that are
separated by some combination of either spaces or commas.
These ``words'' may be alphabetic, numeric, or a combination of both and
represent the command names and their respective arguments.
Command names in \wip\ are case insensitive (\ie commands may be entered in
either lower case, upper case,
or a mixture of both).
Arguments to a command that will be displayed
(\eg a string of characters that will appear in a
label\index{Labels!case sensitive}) does, however, remain case sensitive.
Command names in \wip\ may also be abbreviated to uniqueness.
If a command typed by the user is not unique, \wip\ will issue a warning
and ignore the ambiguous command.
The exception to this rule is exact matches;
a complete command name will always match.

As an example of how to type commands in interactive mode, suppose a
user wanted to draw a single line from one position to another.
The following listing illustrates the commands the user would type
and an example of \wip's response to an ambiguously stated command:
\begin{wiplist}%
  \index{Commands!{\tt move}}
  \index{Commands!{\tt draw}}
  \index{Commands!{\tt end}}
  \index{Commands!{\tt environment}}
  \item {\tt move 0 0}
\samepage
  \item {\tt draw 1 1}
  \item {\tt en}
  \item [\ ] {\tt Ambiguous command name, choices are:}
  \item [\ ] {\tt end ~~~ environment}
  \item {\tt end}
\end{wiplist}

Some of \wip's commands have a number of arguments which must be specified.
If a command requires a certain number of arguments to function properly
and an insufficient number of arguments are given with the command,
then \wip\ will issue a warning to the user.
For example, if the user typed the {\tt move} command without the two
required arguments, \wip\ would warn the user:
\begin{wiplist}%
  \item {\tt move}
\samepage
  \item [\ ] {\tt Insufficient number of arguments provided. 2 required.}
\end{wiplist}
Other commands, however, have optional parameters which will operate
with default values if they are not supplied by the user.
In other words, a command may have a number of arguments that may be
optionally left for the program to determine.
Optional parameters allow the user to alter the default action
of these commands.
For example, the command {\tt limits} has four arguments
that specify the extreme values for the
world coordinates.\index{Coordinate System!World}\index{World Units}
If the user issues the command with no arguments,
\wip\ will determine an acceptable range based on the data currently
loaded into the {\bf X} and {\bf Y} arrays.

We have already introduced the main \wip\ prompt (\wipp).
There are two other prompts in \wip:  \wipd\ and \wipi.
These prompts appear in response to the commands
{\tt define}\index{Commands!{\tt define}}
and {\tt insert}\index{Commands!{\tt insert}},
respectively, and are discussed further in Section~\ref{s:macedit}.

Comments\index{Comments}
may be entered along with \wip\ commands.
The comment character is
the pound sign (\#)\index{Comments!comment character(\#)}
and everything from this
character to the end of the input line is ignored by \wip.
One exception to this are commands involving labels (see Section~\ref{s:text}).
Since there is no easy way for \wip\ to determine whether the comment
character is part of the string or not,
it generally is assumed to be part of the text and is passed
to the command as part of the text to label.
Hence, comment characters should not be used with these commands.
In most of the examples shown in this manual, comments are used to
describe the command or specify arguments in the definition of macros.
If you are trying these examples interactively, you need not type the
comment character nor any of the text that follows it.

Last, but certainly not least, \wip\ provides the user with three
different types of on-line help\index{Help}.
See Section~\ref{s:help} for a description of each of these types of help.
