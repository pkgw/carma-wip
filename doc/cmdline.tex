%% This file is to be included by latex in wip.tex
%
% The Command Line Interface
%
\mylabel{a:cmdline}
\myfile{cmdline.tex}

\wip\ can always be run by specifying
parameters and plot files on the command line.
The command line interface is quite appropriate for
command procedures and shell scripts.\footnote{If you are not
familiar with these terms, don't worry,
they refer to advanced VMS and UNIX concepts you can learn about.
Check with someone more experienced than you or
with a system manager to learn more.}
It is also very useful for getting a plot command file loaded into \wip.

The format for calling \wip\ from the command line is
\begin{wiplist}%
  \item [\% ] {\tt wip  [optionalArgs] [plotfile \ldots ] [-e wipcmd]}
\end{wiplist}
where {\tt optionalArgs} includes the following options:
\begin{wiplist}%
  \item [\ ] {\tt [-d device] [--] [-x] [-r] [-b] [-g] [-q] [-h] [-?]}
\end{wiplist}
Except for the {\tt -e} option (discussed below),
the \wip\ options are, by convention,
placed before any of the plot file names.
The options may, however,
appear anywhere on the command line (after the \wip\ command);
but they are {\em always} evaluated before reading any of the plot files.

All items enclosed in square brackets ([\ldots ]) are optional.
The following list explains the command line arguments and
what function they perform.
\begin{description}
\index{Command Line!switches}
  \item [{\tt -d {\em device-type}} --] Sets the initial graphics
    device\index{Devices} to {\em device-type}.
    This option overrides the default device {\em and}
    any device specified in the user's \wipinit\index{wipinit@\wipinit} file.
    See Section~\ref{s:devices} on how to specify the {\em device-type}.
  \item [{\tt --} --]  Specifies {\tt stdin} as an input plot file.
    This may have some use when using \wip\ in a script.  Note that this
    option also forces the {\tt -x} option because input ends when all
    of data from stdin has arrived.
  \item [{\tt -x} --]  Effectively ``turns off'' the interactive
    state of \wip.
    After \wip\ has processed all of the plot files on the command line,
    a close plot command is issued and \wip\ exits.
    If the current graphics device is a hardcopy\index{Hardcopy} device,
    then the {\tt hardcopy}\index{Commands!{\tt hardcopy}}
    \index{Hardcopy!commands!{\tt hardcopy}}
    command is issued when the last plot file has finished.
  \item [{\tt -r} --]  If \wip\ has been compiled with the {\tt READLINE}
    library, then each time a user exits from \wip,
    a history file is written which contains all of the commands typed.
    This history file is read in by
    the {\tt READLINE} routines when a user enters \wip\ again and the
    commands are available for recalling, editing, and executing.
    Use of this option suppresses the writing of the history file when
    the user exits from \wip.
    Any previous commands will still be available for recall and the
    {\tt READLINE} editing capabilities will still exist.
  \item [{\tt -b} --]  This option completely disables the {\tt READLINE}
    capabilities.  This option may be useful for terminals that do not
    permit the escape sequences to be passed to the {\tt READLINE} routines.
  \item [{\tt -g} --]  Opens \wip\ in debug mode.  This is not generally
    useful to the user but might provide some additional comments in
    cases where the usual diagnostic messages don't help.
  \item [{\tt -q} --]  Turns off all informational messages from \wip.
    This is useful when the selected output file for the device is {\tt stdout}.
    This option is disabled, however, when \wip\ enters interactive mode.
  \item [{\tt -h} or {\tt -?} --]  Writes a usage statement listing all
    of the command line options available within \wip\ and then
    immediately exits.  No plotting is done.
\end{description}

The remaining optional arguments to the \wip\ command
(up to the optional {\tt -e} flag) are the names of either
macro\index{Macros} or plot command files\index{Command Line!input files}.
These files are only
read\footnote{Files on the command line are read in the order in which
they are typed on the command line (left to right) and are
read until either an error occurs or until the interactive state terminates
(an {\tt end} command not associated with either a
{\tt define} or {\tt insert} command).}\index{Commands!{\tt end}}
\index{Commands!{\tt define}}\index{Macros!commands!{\tt define}}%
\index{Commands!{\tt insert}}\index{Macros!commands!{\tt insert}}%
after the \wipinit\index{wipinit@\wipinit}
file has been read, the command line arguments set, and the initial device
(if any) has been opened.
The remaining arguments are then processed just as if the user had
typed the command lines interactively.
When the last file is finished being read, \wip\ moves into interactive
mode (unless the ``{\tt -x}'' option is used).
All macros defined in the input files are now defined in \wip\ and all plot
commands are present in the command buffer\index{Macros!buffer}.

As an example of how to use the command line arguments, consider the
following command:
\begin{wiplist}%
  \item [\%] {\tt wip -x -d myfile.ps/vps myfile.wip}
\end{wiplist}
The file {\tt myfile.wip} is a simple ASCII text file of commands
(similar to the listings that appear in this manual
or those saved with the {\tt write}\index{Commands!{\tt write}} command).
This type of file is a useful way to insert a plot command listing,
for playing back commands (say to another device),
or for defining commonly used macros.
The argument {\tt -d myfile.ps/vps} is the command line
switch that \wip\ uses to redefine the initial device.
This example illustrates how to specify a file name ({\tt myfile.ps})
and associate with a hardcopy device ({\tt /vps}).
For more details on graphical devices, see Section~\ref{s:devices}.
Finally, the {\tt -x} flag tells \wip\ to exit when it is finished
reading all of the command files on the command line
(in this case, just the file {\tt myfile.wip}).

Therefore, this command line example illustrates a quick way to spool a plot off
to the printer without having to enter the interactive mode of \wip.
This might be useful, perhaps, when putting together a paper
where there are figures that will be used are generated at different
times in the analysis stage.
The plot commands can be saved
to a file (like {\tt myfile.wip}) as they are created.
And then later, when the paper is ready to go out,
they can be easily spooled off to the printer.
Users familiar with Unix shell scripts will discover that
this option can be quite useful (and welcome).

Finally, if the {\tt -e} optional flag is present, it signals that the
next and any remaining arguments are a single command to be issued inside \wip.
That command can be any internal \wip\ command or a defined macro.
The command is executed after all of the files on the command line have
been loaded.
If \wip\ is to be run interactively, this command is executed just
before \wip\ enters interactive mode.

The usefulness of this option is seen as a generalization of the example
above.
Suppose the file {\tt myfile.wip} defined a plot but required a file
name as an argument.
Rather than editing {\tt myfile.wip} for each file, \wip\ can be called
with the {\tt -e} option and each file can be passed on the command line.
For example, if the file {\tt myfile.wip} defines a macro called
{\tt dojob}, then multiple files could be input as follows:
\begin{wiplist}%
  \item [\%] {\tt wip -x -d file1.ps/vps myfile.wip -e dojob file1}
  \item [\%] {\tt wip -x -d file2.ps/vps myfile.wip -e dojob file2}
\end{wiplist}
