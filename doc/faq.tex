%% This file is to be included by latex in wip.tex
%
% FAQ.
%
\mylabel{a:faq}
\myfile{faq.tex}

This Appendix presents many frequently asked questions about \wip\ and
\wip\ commands.
These questions have come from users from their usage of \wip.
Hopefully, this list will address most of the questions
that novice users may encounter and present solutions to problems that
others may find useful as they become more experienced and attempt more
complicated plots.
Each topic will first present the problem or question and then a
solution will begin following the `$\Diamond$' character.
This list of questions should also be the most volatile of the entire
manual as more users submit questions and solutions to various problems.

\subsubsection*         {Lack of Printer Output}
\mylabel{sss:hardcopy}

When I run my plot file on my screen, everything looks great.
But when I send it to the printer, nothing shows up.
Here are the commands I execute:
\begin{wiplist}%
  \index{Commands!{\tt device}}
  \index{Commands!{\tt input}}
  \index{Commands!{\tt end}}
  \item [\%] {\tt wip}
\samepage
  \item {\tt input myfile.wip}\hfill\# Plots okay on the screen.
  \item {\tt device myfile.ps/vps}\hfill\# Get a printing device.
  \item {\tt input myfile.wip}\hfill\# Play through the commands again.
  \item {\tt end}\hfill\# All finished.
  \item [\%] {\tt lpr myfile.ps}
\end{wiplist}
The light on the printer flashes for a few minutes and then stops and
does not produce a plot.
What's wrong?

\indent$\Diamond$
Whenever you send a plot to a printing device, you should
terminate the plot by using either the
{\tt hardcopy}\index{Commands!{\tt hardcopy}} command or
another call to the {\tt device}\index{Commands!{\tt device}} command.
These commands inform
the plot that there is not going to be any further information
coming and it should
insert device specific closing commands.  In the case of the
printing device `/vps' shown above, the commands that are
inserted instruct the printer to print and eject the page it has
been generating.
Without these inserted commands,
the plot file is (internally) read by the printer and then ignored
because it is never told to eject the page.

An alternative to the example above is to use
the {\tt phard}\index{Commands!{\tt phard}} command.
This command initiates a (temporarily) new, user specified device
and then plays back all the commands issued and then performs
the {\tt hardcopy} command to close the plot file.  The final
operation of the {\tt phard} command is to reinstate the device
that was active when this command was called.  To illustrate,
the commands shown previously, can be replaced with the following:
\begin{wiplist}%
  \index{Commands!{\tt input}}
  \index{Commands!{\tt phard}}
  \index{Commands!{\tt end}}
  \item [\%] {\tt wip}
\samepage
  \item {\tt input myfile.wip}\hfill\# Plots okay on the screen.
  \item {\tt phard myfile.ps/vps}\hfill\# Spool it to the printing device.
  \item {\tt end}\hfill\# All finished.
\end{wiplist}
If the {\tt print} user variable has its default value,
the plot will automatically be spooled to the printer.

Finally, this plot could also be sent by using the single command line:
\begin{wiplist}%
  \item [\%] {\tt wip -x -d myfile.ps/vps myfile.wip}
\end{wiplist}
See Appendix~\ref{a:cmdline} for an explanation of the command
line options and their meanings.

\subsubsection*         {{\tt erase} Command Troubles}
\mylabel{sss:erasehard}

I have a macro defined as follows:
\begin{wiplist}%
  \index{Commands!{\tt define}}
  \index{Commands!{\tt erase}}
  \index{Commands!{\tt expand}}
  \index{Commands!{\tt set}}
  \index{Commands!{\tt loop}}
  \index{Commands!{\tt end}}
  \item {\tt define doloop}\hfill\# \$1=Passed counter.
\samepage
  \item [\wipd] {\tt erase}\hfill\# Erase the screen.
  \item [\wipd] {\tt expand 0.5}\hfill\# Initialize the expansion.
  \item [\wipd] {\tt set \esc{0} -1}\hfill\# Initialize a counter.
  \item [\wipd] {\tt loop 16 mymacro \esc{0} \$1}\hfill\# Call another macro.
  \item [\wipd] {\tt end}\hfill\# End of macro definition.
\end{wiplist}
which I use to plot multi-page panel plots
({\tt panel}\index{Commands!{\tt panel}} commands appear
in {\tt mymacro} macro).
When I call this macro as follows:
\begin{wiplist}%
  \index{Commands!{\tt device}}
  \index{Commands!{\tt hardcopy}}
  \item {\tt device myfile.ps/vps}\hfill\# Set up the printer file.
\samepage
  \item {\tt doloop 1}\hfill\# Do the first page.
  \item {\tt hardcopy}\hfill\# Send it to the printer.
  \item {\tt doloop 2}\hfill\# Do the second page.
  \item {\tt hardcopy}\hfill\# Send it to the printer.
  \item {\tt doloop 3}\hfill\# Do the third page.
  \item {\tt hardcopy}\hfill\# Send it to the printer.
\end{wiplist}
I get the following error messages:
\begin{wiplist}%
  \item [\ ] {\tt Trouble setting expand value.}
\samepage
  \item [\ ] {\tt Saving expand as 1.000000.}
  \item [\ ] {\tt Trouble setting expand value.}
  \item [\ ] {\tt Saving expand as 1.000000.}
\end{wiplist}
I am calling {\tt expand} with a proper argument; what is wrong?

\indent$\Diamond$
The error message is not so much to do with the macro, per say, as
to what it is going on in the macro.  Every time the command
{\tt hardcopy} is called,
the device that was used to do the drawing is closed.
However, the
next command to be executed is to enter the macro again and the first
"graphical command" is to set the expansion size.
\wip\ checks that the requested expand size matches the returned value
from PGPLOT.
If the two do not match, a warning message is sent
(which is the message that was displayed).
Now, PGPLOT interprets {\tt erase} commands as a new page request.
For printing devices, this is basically the same as the {\tt hardcopy}
command except that it does not close the device.
Because the macro doloop above
has the {\tt erase} command at the top of it,
a suggested alternative to the commands above would be
(the macros may stay the same):
\begin{wiplist}%
  \index{Commands!{\tt device}}
  \index{Commands!{\tt hardcopy}}
  \item {\tt device myfile.ps/vps}\hfill\# Set up the printer file.
\samepage
  \item {\tt doloop 1}\hfill\# Do the first page.
  \item {\tt doloop 2}\hfill\# Do the second page.
  \item {\tt doloop 3}\hfill\# Do the third page.
  \item {\tt hardcopy}\hfill\# Send it to the printer.
\end{wiplist}
The erase command will automatically switch to a new page of paper for
hardcopy (printing) devices (and all drawing will appear in the same file)
or erase the screen for interactive devices.

\subsubsection*         {Offset Positions}
\mylabel{sss:offsets}

I often need an offset to move away from my current position.
Is there (or can there be) a command like ``shift'' that moves
the pointer relative to the previous move or draw end position?

\indent$\Diamond$
A general offset command would be quite difficult to create due
to the possible range and direction of world coordinates.
If you want a macro that would do it for a particular orientation
and scale (\eg RA/dec plots), consider the following macro definition:
\begin{wiplist}%
  \index{Commands!{\tt define}}
  \index{Commands!{\tt new}}
  \index{Commands!{\tt set}}
  \index{Commands!{\tt move}}
  \index{Commands!{\tt free}}
  \index{Commands!{\tt end}}
  \index{User Variables!commands!{\tt new}}
  \index{User Variables!commands!{\tt free}}
  \item {\tt define doshift}\hfill\# \$1=delx; \$2=dely.
\samepage
  \item [\wipd] {\tt new offx offy}\hfill\# Creates offx and offy.
  \item [\wipd] {\tt set offx cx + \$1}\hfill\# Set up $x$-offset.
  \item [\wipd] {\tt set offy cy + \$2}\hfill\# Set up $y$-offset.
  \item [\wipd] {\tt move offx offy}\hfill\# Move to the offset position.
  \item [\wipd] {\tt free offx offy}\hfill\# Release offx and offy.
  \item [\wipd] {\tt end}
\end{wiplist}
and commands:
\begin{wiplist}%
  \index{Commands!{\tt move}}
  \index{Commands!{\tt draw}}
  \item {\tt move 5000 50000}\hfill\# Some command that sets position.
\samepage
  \item {\tt doshift 30 300}\hfill\# Shift the position.
  \item {\tt draw 5100 49000}\hfill\# Another set position command.
  \item {\tt doshift 30 500}\hfill\# Shift the position.
\end{wiplist}

The {\tt move} or {\tt draw} commands could be folded into
the {\tt doshift} macro by adding another 3 arguments, as in:
\begin{wiplist}%
  \index{Commands!{\tt define}}
  \index{Commands!{\tt new}}
  \index{Commands!{\tt move}}
  \index{Commands!{\tt draw}}
  \index{Commands!{\tt set}}
  \index{Commands!{\tt free}}
  \index{Commands!{\tt end}}
  \index{User Variables!commands!{\tt new}}
  \index{User Variables!commands!{\tt free}}
  \item {\tt define mdshift}\hfill\# \$1/2=delx/y; \$3/4=x/y; \$5=m/d.
\samepage
  \item [\wipd] {\tt new offx offy}\hfill\# Creates offx and offy.
  \item [\wipd] {\tt if (0\$5 == 0) move \$3 \$4}\hfill\# Move if (\$5 == 0).
  \item [\wipd] {\tt if (0\$5 == 1) draw \$3 \$4}\hfill\# Draw if (\$5 == 1).
  \item [\wipd] {\tt set offx cx + \$1}\hfill\# Set up $x$-offset.
  \item [\wipd] {\tt set offy cy + \$2}\hfill\# Set up $y$-offset.
  \item [\wipd] {\tt move offx offy}\hfill\# Move to the offset position.
  \item [\wipd] {\tt free offx offy}\hfill\# Release offx and offy.
  \item [\wipd] {\tt end}
\end{wiplist}
Then, you can use it as:
\begin{wiplist}%
  \item {\tt mdshift 30 300 5000 50000 0}
\samepage
  \item {\tt mdshift 30 500 5100 49000 1}
\end{wiplist}

\subsubsection*         {Writing Numbers as Text}
\mylabel{sss:textnumbers}

I have been using \wip\ to make a macro for doing multi-panel
plots with velocity labeling all automatically grabbed
from the image header, or at least attempting to do so.
I have made good progress but I am having trouble dealing with
changing numbers into characters and having the various text
writing command believe in text variables.\index{Labels!user variables in}
For example if I calculate the velocity of a channel in a macro,
how do I load that value into a text variable and get
it printed to the screen without doing it one symbol at a time?

\indent$\Diamond$
\begin{verbatim}
set \12 (295 + (\11 * 5))
echo \12
new chv
set chv \12
mtext B -1 0.1 0.0 V\dlsr\u = \[chv]
free chv
\end{verbatim}
where register 11 is the channel number which is changed
in some loop as we step through the channels.

% This is also illustrated in section~\ref{p:symbol}.  %% check - 

\subsubsection*         {Box Label is Incorrect}
\mylabel{sss:boxy}

I am trying to display an image with Right Ascension and
Declination coordinates.  When I use the command
\begin{wiplist}%
  \index{Commands!{\tt box}}
  \item {\tt box bcnstz bcnstvz}
\end{wiplist}
with my image,
the $y$-coordinate values appear incorrect.
The {\tt header}\index{Commands!{\tt header}} command appears to be 
getting the parameters okay
(using the `{\tt show i}'\index{Commands!{\tt show}} command).

\indent$\Diamond$
What is happening is that the second argument to the {\tt box} command
is missing the {\tt y} character.
This character tells the time labeling routine not to break up the label
at the 24 hour mark (\ie do not do any day formatting).
To correctly display declination, you must include both the {\tt y} and
the {\tt z} character in the second argument to {\tt box}.
Another alternative is to include in the second argument the {\tt d} option
(and the {\tt h} in the first argument)
which will add unit superscripts.
The {\tt d} option implicitly implies the {\tt y} option.
For more details about these and other characters and what duties they perform,
see Table~\ref{t:box}.

\subsubsection*         {FITS Images}
\mylabel{sss:fitsimages}

I have loaded a FITS\index{FITS} image,
retrieved the header, and displayed and boxed the image in \wip\ as follows:
\begin{wiplist}%
  \index{Commands!{\tt image}}
  \index{Commands!{\tt header}}
  \index{Commands!{\tt halftone}}
  \index{Commands!{\tt box}}
  \item {\tt image myimage.fits}
\samepage
  \item {\tt header rd}
  \item {\tt halftone}
  \item {\tt box bcnsthz bcnstvdzy}
\end{wiplist}
No halftone is present and the box is label incorrectly.
I have displayed this image in IRAF before making a FITS file
with no problems; why can't \wip\ display it?

\indent$\Diamond$
Some FITS image writers omit keywords.
Make sure that the image header keywords required by \wip\ are present
in the FITS image.
In order to properly read header information and use it to display
the coordinates with the {\tt box} command, the optional header keywords
should also be present.
See the FITS image description in Section~\ref{s:imfile} for a list of
required and requested keywords.
Also see Section~\ref{p:image} for a sample FITS header.

\subsubsection*         {Fortran Callable}
\mylabel{sss:fortran}

Is it possible to call various \wip\ routines from a Fortran program?

\indent$\Diamond$
No.
\wip\ is a collection of routines designed to act as a user interface to
the PGPLOT graphics routines.
Fortran programmers should call the individual PGPLOT routines directly.

\subsubsection*         {Panel Movements with Images}
\mylabel{sss:paneladjust}

I am using the {\tt panel} command to display multiple images side by side.
I want to have the individual boxes butt up against each other.
However, every time I call {\tt panel} with negative arguments (even
after calling {\tt submargin 0 0}),
there is still a gap between the boxes.

\indent$\Diamond$
This is dealt with in Section~\ref{s:imwinadj}.

\subsubsection*         {User Interaction with Prompts}
\mylabel{sss:prompting}

This problem concerns writing instructions to users inside
the \wip\ command window.
For example, a macro
allows the user to select where in a frame
to place a label using the cursor.
But, if this macro is to be generally usable,
it really should write to the
command window a message telling the user what the macro is expecting at
each cursor prompt.
Can this be done simply in \wip ?

\indent$\Diamond$
This is quite easy to do with the
{\tt echo}\index{Commands!{\tt echo}}
command.
The following simple macro shows how to prompt just before a cursor call.
\begin{wiplist}%
  \index{Commands!{\tt define}}
  \index{Macros!commands!{\tt define}}
  \index{Commands!{\tt echo}}
  \index{Commands!{\tt cursor}}
  \index{Commands!{\tt label}}
  \index{Commands!{\tt end}}
  \index{Macros!commands!{\tt end}}
  \item {\tt define doprompt}\hfill\# \$1-\$0 used for the label.
\samepage
  \item [\wipd] {\tt echo "Find the desired position"}\hfill\# Prompt the user.
  \item [\wipd] {\tt echo "and then hit any key."}
  \item [\wipd] {\tt cursor}\hfill\# Move to the desired position.
  \item [\wipd] \hfill\# Put the label at the cursor position.
  \item [\wipd] {\tt label \$1 \$2 \$3 \$4 \$5 \$6 \$7 \$8 \$9 \$0}
  \item [\wipd] {\tt end}\hfill\# End of macro definition.
\end{wiplist}

\subsubsection*         {Overlapping Y-Axis Labels}
\mylabel{sss:vertlabs}

When I chose vertical tick labels along the $y$-axis of the box,
the $y$-axis text label overlaps them (especially if the numbers
at a major tick interval is 3 or more digits).

\indent$\Diamond$
Vertical labeling is a problem for large value coordinates (like declination).
The default {\tt ylabel}\index{Commands!{\tt ylabel}} command is
actually an alias for the {\tt mtext}\index{Commands!{\tt mtext}} command.
In fact, the following two commands are identical:
\begin{wiplist}%
  \item [\wipd] {\tt mtext L 2.2 0.5 0.5 Yaxis label string}
\samepage
  \item [\wipd] {\tt ylabel Yaxis label string}
\end{wiplist}
To use labels along the $y$-axis without overlapping the tick labels,
use the {\tt mtext} command with a larger displacement (\ie use 3 or 4
instead of 2.2).
Also be careful to make sure that the left viewport value is large enough
that it does not clip the tick labels.

\subsubsection*         {Logarithmic Error Bars}
\mylabel{sss:logerrs}

I want to plot error bars on a log plot.
Is there a simple way to do this?
I've tried using the {\tt logarithm err} command but can't seem to get
it to work.
Any suggestions?

\indent$\Diamond$
Logarithmic plots with error bars is a bit tricky because the values in
the {\bf ERR} array are offset values rather than the actual error values.
To solve this, either modify the data outside of \wip\ or write
a macro to do the math internally.
Examples of both are shown below.
In both examples, it is assumed that the {\bf X, Y}, and {\bf ERR}
data in {\tt data.lis} are found in columns 1, 2, and 3, respectively.
For UNIX users, the {\tt awk} command:
\begin{wiplist}%
  \item [\%] {\tt awk '\{print \$0,(\$2+\$3)/\$2,\$2/(\$2-\$3)\}' data.lis $>$ d2.lis}
\end{wiplist}
will generate a new file called {\tt d2.lis} which has the same
contents as the old file plus new values in the fourth and fifth columns.
Using {\tt d2.lis}, the following \wip\ commands should work:
\begin{wiplist}%
  \index{Commands!{\tt xcolumn}}
  \index{Commands!{\tt ycolumn}}
  \index{Commands!{\tt ecolumn}}
  \index{Commands!{\tt logarithm}}
  \index{Commands!{\tt points}}
  \index{Commands!{\tt errorbar}}
  \item {\tt data d2.lis}
  \samepage
  \item {\tt xcolumn 1}
  \item {\tt ycolumn 2}
  \item {\tt ecolumn 4}
  \item {\tt logarithm y}
  \item {\tt logarithm err}
  \item {\tt points}
  \item {\tt errorbar 2}
  \item {\tt ecolumn 5}
  \item {\tt logarithm err}
  \item {\tt errorbar 4}
\end{wiplist}
or, within \wip, one can use:
\begin{wiplist}%
  \index{Commands!{\tt define}}
  \index{Commands!{\tt set}}
  \index{Commands!{\tt xcolumn}}
  \index{Commands!{\tt ycolumn}}
  \index{Commands!{\tt ecolumn}}
  \index{Commands!{\tt logarithm}}
  \index{Commands!{\tt points}}
  \index{Commands!{\tt errorbar}}
  \index{Commands!{\tt loop}}
  \item {\tt define eplus}
  \item [\wipd] {\tt set \$1 \$1 + 1}
  \item [\wipd] {\tt set err[\$1] log((y[\$1] + err[\$1]) / y[\$1])}
  \item [\wipd] {\tt end}
  \item {\tt define eminus}
  \item [\wipd] {\tt set \$1 \$1 + 1}
  \item [\wipd] {\tt set err[\$1] log(y[\$1] / (y[\$1] - err[\$1]))}
  \item [\wipd] {\tt end}
  \item {\tt data data.lis}
  \item {\tt xcolumn 1}
  \item {\tt ycolumn 2}
  \item {\tt ecolumn 3}
  \item {\tt set \esc{0} 0}
  \item {\tt loop npts(err) eplus \esc{0}}
  \item {\tt logarithm y}
  \item {\tt points}
  \item {\tt errorbar 2}
  \item {\tt ycolumn 2}
  \item {\tt ecolumn 3}
  \item {\tt set \esc{0} 0}
  \item {\tt loop npts(err) eminus \esc{0}}
  \item {\tt logarithm y}
  \item {\tt points}
  \item {\tt errorbar 4}
\end{wiplist}

\subsubsection*         {Aspect Ratio Without Images}
\mylabel{sss:aspect}

I have tried to plot the spatial distribution of stars
(\ie dots on Right Ascension -- Declination plot)
but I can't get the aspect ratio of the map right.
The star positions are read in from a text file.
I know that {\tt winadj}\index{Commands!{\tt winadj}}%
\index{Coordinate System!commands!{\tt winadj}}
works for images but when I use it for a simple plot with individual points,
it didn't seem to understand that the Right Ascension unit
is 15 times smaller than the Declination unit. 
Am I missing something?

\indent$\Diamond$
In order to get the spatial distribution correct, you will have to
help \wip\ out.
You are correct that \wip\ doesn't know that Right Ascension is about
15 times smaller than the Declination axis.
This is because the world coordinate
system\index{World Units}\index{Coordinate System!World}
has no units associated with it.
The best solution is use {\tt winadj} with the spatial
extent arguments provided in the same units.
For example, if your plot
runs in Right Ascension from $5^h 37^m$ to $5^h 29^m$
and Declination from $-5^\circ$ to $-8^\circ$, you might use the following:
\begin{wiplist}%
  \index{Commands!{\tt set}}
  \index{Commands!{\tt winadj}}
  \index{Commands!{\tt limits}}
  \index{Commands!{\tt box}}
  \index{User Variables!commands!{\tt set}}
  \index{Coordinate System!commands!{\tt limits}}
  \index{Coordinate System!commands!{\tt winadj}}
  \index{Boxes!commands!{\tt box}}
  \item {\tt set \esc{1} (5 * 3600) + (37 * 60)}
  \item {\tt set \esc{2} (5 * 3600) + (29 * 60)}
  \item {\tt set \esc{3} -8 * 3600}
  \item {\tt set \esc{4} -5 * 3600}
  \item {\tt set \esc{5} (\esc{1} - \esc{2}) * 15}
  \item {\tt set \esc{6} \esc{4} - \esc{3}}
  \item {\tt winadj 0 \esc{5} 0 \esc{6}}
  \item {\tt limits \esc{1} \esc{2} \esc{3} \esc{4}}
  \item {\tt box bcnsthz bcnstvdzy}
\end{wiplist}
This will generate the proper 2:3 aspect ratio and also label the coordinates
correctly.

\subsubsection*         {Strange Hardcopy Output}
\mylabel{sss:badcopy}

Suppose I have two \wip\ plotting command files and do the following:
\begin{wiplist}%
  \index{Commands!{\tt input}}
  \index{Commands!{\tt phard}}
  \index{Hardcopy!commands!{\tt phard}}
  \item {\tt input first.wip}\hfill\# Plots out OK on the workstation.
  \samepage
  \item {\tt phard /ps}\hfill\# Plots out OK on the PostScript printer.
  \item {\tt input second.wip}\hfill\# Plots out OK on the workstation.
  \item {\tt phard /ps}\hfill\# Plots both first and second figures!
\end{wiplist}
Why does the second PostScript file contain both figures?

\indent$\Diamond$
What you are actually getting out in the second PostScript file
is the combination of both {\tt input} commands.
This is because the {\tt input} command stays on the playback
stack and the {\tt phard} command plays back all commands
on the stack.
Hence, the second {\tt phard} command will call {\tt input first.wip}
followed by {\tt input second.wip}.

\subsubsection*         {Changing the Number of Digits Displayed}
\mylabel{sss:nsig}

I wish to do something quite simple: have a number like 1.234
(which is generated within a \wip\ macro) be part of a plot label.
Is this possible?
If so, there a way to control the number of significant
figures for this purpose?

\indent$\Diamond$
The answer to both questions is yes.
Any label with the construct {\tt \esc{[{\em x}]}}
will actually be plotted as if the current value of the user variable {\em x}
was used in the command.
See Section~\ref{s:text} and Table~\ref{t:text} for more details on how
to insert user variables in a text command.

To control the number of significant digits,
change the value of the user variable {\tt nsig}.
By default, it is set to 3;
specifying a larger value will cause more digits to be displayed.

\subsubsection*         {Leaving Loops Early}
\mylabel{sss:loops}

Is it possible to break out of a {\tt loop} command early
(\ie before the total number of iterations are reached)
based on some condition within that macro?

\indent$\Diamond$
The only way to do this is (not very elegant) to add a
conditional to the macro and have the macro exit when
the condition is met.
This will still mean the macro is called the requested number of times,
but each time it is called with the counter greater than the
conditional value, it will just exit.
A trivial example follows that will only print out the first ten numbers
even though the loop is being executed twenty times:
\begin{wiplist}%
  \index{Commands!{\tt define}}
  \index{Commands!{\tt set}}
  \index{Commands!{\tt if}}
  \index{Commands!{\tt echo}}
  \index{Commands!{\tt end}}
  \index{Commands!{\tt loop}}
  \index{Macros!commands!{\tt define}}
  \index{Macros!commands!{\tt end}}
  \index{Macros!commands!{\tt if}}
  \index{Macros!commands!{\tt loop}}
  \index{User Variables!commands!{\tt set}}
  \item {\tt define dummy}\hfill\# \$1=counter.
  \samepage
  \item [\wipd] {\tt set \$1 \$1 + 1}\hfill\# Increment the counter.
  \item [\wipd] {\tt if (\$1 $>$ 10) end}\hfill\# Exit if condition is true.
  \item [\wipd] {\tt echo \$1}\hfill\# Write out the current value.
  \item [\wipd] {\tt end}\hfill\#  End the macro definition.
  \item {\tt set \esc{0} 0}\hfill\# Initialize the counter.
  \item {\tt loop 20 dummy \esc{0}}\hfill\# Loop through the macro.
\end{wiplist}


\subsubsection*         {Printing labels next to points using loops}

It is perhaps not immediately obvious how to print a text label \"next\"
to a point using WIP. The {\tt loop} command can also be used for this,
but at the cost of having to know the number of points.

\begin{verbatim}

# Labels points in a file
# $1 = column number in datafile # $2 = x offset
# $3 = y offset
# $4 = expand factor for charactor size
define labelthem
set \0 \0 + 1
new "var"
lines \0 \0
str var $1 $1
set \1 x[\0] + $2
set \2 y[\0] + $3
move \1 \2
expand $4
label \[var]
free var
expand 1
end


data datafile
xcol 1
ycol 2
points
loop 30 labelthem 3 0.03 0. 0.6


\end{verbatim}
